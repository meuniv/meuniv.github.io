%% Generated by Sphinx.
\def\sphinxdocclass{report}
\documentclass[letterpaper,10pt,english]{sphinxmanual}
\ifdefined\pdfpxdimen
   \let\sphinxpxdimen\pdfpxdimen\else\newdimen\sphinxpxdimen
\fi \sphinxpxdimen=.75bp\relax

\PassOptionsToPackage{warn}{textcomp}
\usepackage[utf8]{inputenc}
\ifdefined\DeclareUnicodeCharacter
% support both utf8 and utf8x syntaxes
  \ifdefined\DeclareUnicodeCharacterAsOptional
    \def\sphinxDUC#1{\DeclareUnicodeCharacter{"#1}}
  \else
    \let\sphinxDUC\DeclareUnicodeCharacter
  \fi
  \sphinxDUC{00A0}{\nobreakspace}
  \sphinxDUC{2500}{\sphinxunichar{2500}}
  \sphinxDUC{2502}{\sphinxunichar{2502}}
  \sphinxDUC{2514}{\sphinxunichar{2514}}
  \sphinxDUC{251C}{\sphinxunichar{251C}}
  \sphinxDUC{2572}{\textbackslash}
\fi
\usepackage{cmap}
\usepackage[T1]{fontenc}
\usepackage{amsmath,amssymb,amstext}
\usepackage{babel}



\usepackage{times}
\expandafter\ifx\csname T@LGR\endcsname\relax
\else
% LGR was declared as font encoding
  \substitutefont{LGR}{\rmdefault}{cmr}
  \substitutefont{LGR}{\sfdefault}{cmss}
  \substitutefont{LGR}{\ttdefault}{cmtt}
\fi
\expandafter\ifx\csname T@X2\endcsname\relax
  \expandafter\ifx\csname T@T2A\endcsname\relax
  \else
  % T2A was declared as font encoding
    \substitutefont{T2A}{\rmdefault}{cmr}
    \substitutefont{T2A}{\sfdefault}{cmss}
    \substitutefont{T2A}{\ttdefault}{cmtt}
  \fi
\else
% X2 was declared as font encoding
  \substitutefont{X2}{\rmdefault}{cmr}
  \substitutefont{X2}{\sfdefault}{cmss}
  \substitutefont{X2}{\ttdefault}{cmtt}
\fi


\usepackage[Bjarne]{fncychap}
\usepackage{sphinx}

\fvset{fontsize=\small}
\usepackage{geometry}


% Include hyperref last.
\usepackage{hyperref}
% Fix anchor placement for figures with captions.
\usepackage{hypcap}% it must be loaded after hyperref.
% Set up styles of URL: it should be placed after hyperref.
\urlstyle{same}
\addto\captionsenglish{\renewcommand{\contentsname}{Contents:}}

\usepackage{sphinxmessages}
\setcounter{tocdepth}{1}



\title{CompThermo}
\date{Nov 20, 2020}
\release{0.0.0.1}
\author{Vincent Meunier}
\newcommand{\sphinxlogo}{\vbox{}}
\renewcommand{\releasename}{Release}
\makeindex
\begin{document}

\pagestyle{empty}
\sphinxmaketitle
\pagestyle{plain}
\sphinxtableofcontents
\pagestyle{normal}
\phantomsection\label{\detokenize{index::doc}}



\chapter{General purpose}
\label{\detokenize{index:general-purpose}}
The purpose of this tool is to convert a \sphinxtitleref{yaml} file into a file usable for
multiple\sphinxhyphen{}choice questions in LMS. The code also creates latex file as needed.


\chapter{Installation}
\label{\detokenize{index:installation}}
There is no installation needed other than making sure that you have a working
Latex environment in place


\chapter{YAML Format}
\label{\detokenize{index:yaml-format}}
This is a simple example:

\begin{sphinxVerbatim}[commandchars=\\\{\}]
\PYGZhy{} Type: MC
  Text: \PYGZgt{}\PYGZhy{}
         In the screencast we derived an expression for the Fermi\PYGZhy{}Dirac distribution
       using the grand canonical ensemble. What is the grand canonical ensemble?
  Size: Auto
    Points: 2
    Answers:
      \PYGZhy{} \PYGZgt{}\PYGZhy{}
        The ensemble of systems with fixed energy and entropy.
      \PYGZhy{} \PYGZgt{}\PYGZhy{}
        The ensemble of systems with fixed energy and chemical potential.
      \PYGZhy{} \PYGZgt{}\PYGZhy{}
        The ensemble of systems with fixed temperature and chemical potential.
    Validity:
      \PYGZhy{} incorrect
      \PYGZhy{} incorrect
      \PYGZhy{} correct
    Quiz: lecture 29
\end{sphinxVerbatim}

We can also write latex formula, something like:

\begin{sphinxVerbatim}[commandchars=\\\{\}]
Since Pythagoras, we know that :math:`a\PYGZca{}2 + b\PYGZca{}2 = c\PYGZca{}2`.
\end{sphinxVerbatim}

to write something like:

Since Pythagoras, we know that \(a^2 + b^2 = c^2\).
\begin{quote}

Or we can write full line using:

\begin{sphinxVerbatim}[commandchars=\\\{\}]
\PYG{o}{.}\PYG{o}{.} \PYG{n}{math}\PYG{p}{:}\PYG{p}{:} \PYG{p}{(}\PYG{n}{a} \PYG{o}{+} \PYG{n}{b}\PYG{p}{)}\PYG{o}{\PYGZca{}}\PYG{l+m+mi}{2} \PYG{o}{=} \PYG{n}{a}\PYG{o}{\PYGZca{}}\PYG{l+m+mi}{2} \PYG{o}{+} \PYG{l+m+mi}{2}\PYG{n}{ab} \PYG{o}{+} \PYG{n}{b}\PYG{o}{\PYGZca{}}\PYG{l+m+mi}{2} \PYGZbs{}\PYG{n}{oint} \PYG{n}{x}\PYG{o}{\PYGZca{}}\PYG{p}{\PYGZob{}}\PYG{n}{i}\PYG{o}{\PYGZca{}}\PYG{n}{h}\PYG{p}{\PYGZcb{}}
\end{sphinxVerbatim}
\end{quote}

To yield:
\begin{quote}
\begin{equation*}
\begin{split}(a + b)^2 = a^2 + 2ab + b^2 \oint x^{i^h}\end{split}
\end{equation*}\end{quote}

This is a simple example:

\begin{sphinxVerbatim}[commandchars=\\\{\}]
\PYG{k+kn}{import} \PYG{n+nn}{math}
\PYG{n+nb}{print} \PYG{l+s+s1}{\PYGZsq{}}\PYG{l+s+s1}{import done}\PYG{l+s+s1}{\PYGZsq{}}
\end{sphinxVerbatim}



\renewcommand{\indexname}{Index}
\printindex
\end{document}