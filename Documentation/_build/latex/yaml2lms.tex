%% Generated by Sphinx.
\def\sphinxdocclass{report}
\documentclass[letterpaper,10pt,english]{sphinxmanual}
\ifdefined\pdfpxdimen
   \let\sphinxpxdimen\pdfpxdimen\else\newdimen\sphinxpxdimen
\fi \sphinxpxdimen=.75bp\relax

\PassOptionsToPackage{warn}{textcomp}
\usepackage[utf8]{inputenc}
\ifdefined\DeclareUnicodeCharacter
% support both utf8 and utf8x syntaxes
  \ifdefined\DeclareUnicodeCharacterAsOptional
    \def\sphinxDUC#1{\DeclareUnicodeCharacter{"#1}}
  \else
    \let\sphinxDUC\DeclareUnicodeCharacter
  \fi
  \sphinxDUC{00A0}{\nobreakspace}
  \sphinxDUC{2500}{\sphinxunichar{2500}}
  \sphinxDUC{2502}{\sphinxunichar{2502}}
  \sphinxDUC{2514}{\sphinxunichar{2514}}
  \sphinxDUC{251C}{\sphinxunichar{251C}}
  \sphinxDUC{2572}{\textbackslash}
\fi
\usepackage{cmap}
\usepackage[T1]{fontenc}
\usepackage{amsmath,amssymb,amstext}
\usepackage{babel}



\usepackage{times}
\expandafter\ifx\csname T@LGR\endcsname\relax
\else
% LGR was declared as font encoding
  \substitutefont{LGR}{\rmdefault}{cmr}
  \substitutefont{LGR}{\sfdefault}{cmss}
  \substitutefont{LGR}{\ttdefault}{cmtt}
\fi
\expandafter\ifx\csname T@X2\endcsname\relax
  \expandafter\ifx\csname T@T2A\endcsname\relax
  \else
  % T2A was declared as font encoding
    \substitutefont{T2A}{\rmdefault}{cmr}
    \substitutefont{T2A}{\sfdefault}{cmss}
    \substitutefont{T2A}{\ttdefault}{cmtt}
  \fi
\else
% X2 was declared as font encoding
  \substitutefont{X2}{\rmdefault}{cmr}
  \substitutefont{X2}{\sfdefault}{cmss}
  \substitutefont{X2}{\ttdefault}{cmtt}
\fi


\usepackage[Bjarne]{fncychap}
\usepackage{sphinx}

\fvset{fontsize=\small}
\usepackage{geometry}


% Include hyperref last.
\usepackage{hyperref}
% Fix anchor placement for figures with captions.
\usepackage{hypcap}% it must be loaded after hyperref.
% Set up styles of URL: it should be placed after hyperref.
\urlstyle{same}
\addto\captionsenglish{\renewcommand{\contentsname}{Contents:}}

\usepackage{sphinxmessages}
\setcounter{tocdepth}{2}



\title{Yaml2LMS}
\date{Nov 21, 2020}
\release{0.0.0.1}
\author{Vincent Meunier}
\newcommand{\sphinxlogo}{\vbox{}}
\renewcommand{\releasename}{Release}
\makeindex
\begin{document}

\pagestyle{empty}
\sphinxmaketitle
\pagestyle{plain}
\sphinxtableofcontents
\pagestyle{normal}
\phantomsection\label{\detokenize{index::doc}}



\chapter{General purpose}
\label{\detokenize{General Purpose:general-purpose}}\label{\detokenize{General Purpose::doc}}
The purpose of this tool is to convert a \sphinxtitleref{yaml} file into a file usable for
multiple\sphinxhyphen{}choice questions in LMS. The code also creates  \(\textrm{\LaTeX}\) and PDF files as needed.


\section{Functionalities}
\label{\detokenize{General Purpose:functionalities}}\begin{enumerate}
\sphinxsetlistlabels{\arabic}{enumi}{enumii}{}{)}%
\item {} 
Creation of files needed for integration with LMS or Canvas

\item {} 
Creation of latex files using the exam class

\item {} 
Creation of latex files with answer keys

\item {} 
Spellchecking

\end{enumerate}


\section{Installation}
\label{\detokenize{General Purpose:installation}}
There is no installation needed as this is a simply Python script.

Just make sure to have:
\begin{enumerate}
\sphinxsetlistlabels{\arabic}{enumi}{enumii}{}{.}%
\item {} 
A working  \(\textrm{\LaTeX}\) environment in place

\item {} 
Python installed

\end{enumerate}

If you use specific Latex packages in your mathematical formulas, you will need to edit the python script directly (work in progress to make this process simpler).


\chapter{Basic usage and configuration file}
\label{\detokenize{usage:basic-usage-and-configuration-file}}\label{\detokenize{usage:config-label}}\label{\detokenize{usage::doc}}
run the python script in a directory where the \sphinxcode{\sphinxupquote{config.yaml}} file described below is located.


\section{Configuration file}
\label{\detokenize{usage:configuration-file}}
All information is provided in the \sphinxcode{\sphinxupquote{config.yaml}} input file. The file reads as:

\begin{sphinxVerbatim}[commandchars=\\\{\}]
\PYG{n+nt}{yamlfile}\PYG{p}{:} \PYG{l+lScalar+lScalarPlain}{quiz20.yaml}
\PYG{n+nt}{spellcheck}\PYG{p}{:} \PYG{l+lScalar+lScalarPlain}{no}
\PYG{n+nt}{createLMS}\PYG{p}{:} \PYG{l+lScalar+lScalarPlain}{yes}
\PYG{n+nt}{createLMS\PYGZus{}text}\PYG{p}{:} \PYG{l+lScalar+lScalarPlain}{no}
\PYG{n+nt}{base}\PYG{p}{:} \PYG{l+s}{\PYGZdq{}}\PYG{l+s}{http://homepages.rpi.edu/\PYGZti{}meuniv/Images/TSM\PYGZus{}F20/}\PYG{l+s}{\PYGZdq{}}
\PYG{n+nt}{dir}\PYG{p}{:} \PYG{l+s}{\PYGZdq{}}\PYG{l+s}{THERMO}\PYG{l+s}{\PYGZdq{}}
\PYG{n+nt}{title1}\PYG{p}{:} \PYG{l+s}{\PYGZsq{}}\PYG{l+s}{PHYS}\PYG{n+nv}{ }\PYG{l+s}{\PYGZob{}4420\PYGZcb{}:}\PYG{n+nv}{ }\PYG{l+s}{Thermodynamics}\PYG{n+nv}{ }\PYG{l+s}{and}\PYG{n+nv}{ }\PYG{l+s}{Statistical}\PYG{n+nv}{ }\PYG{l+s}{Mechanics}\PYG{n+nv}{ }\PYG{l+s}{(Quiz}\PYG{n+nv}{ }\PYG{l+s}{20)}\PYG{l+s}{\PYGZsq{}}
\PYG{n+nt}{title2}\PYG{p}{:} \PYG{l+s}{\PYGZdq{}}\PYG{l+s}{Dr.}\PYG{n+nv}{ }\PYG{l+s}{Vincent}\PYG{n+nv}{ }\PYG{l+s}{Meunier,}\PYG{n+nv}{ }\PYG{l+s}{Fall}\PYG{n+nv}{ }\PYG{l+s}{2020}\PYG{l+s}{\PYGZdq{}}
\PYG{n+nt}{solutionKey}\PYG{p}{:} \PYG{l+lScalar+lScalarPlain}{yes}
\end{sphinxVerbatim}

The various keywords (ending with a \sphinxcode{\sphinxupquote{:}}) are mostly self\sphinxhyphen{}explanatory.
\begin{enumerate}
\sphinxsetlistlabels{\arabic}{enumi}{enumii}{}{.}%
\item {} 
\sphinxcode{\sphinxupquote{yamlfile}} provides the actual file with the list of questions (see here: {\hyperref[\detokenize{Question format:my-reference-label}]{\sphinxcrossref{\DUrole{std,std-ref}{YAML Format (input)}}}})

\item {} 
\sphinxcode{\sphinxupquote{spellcheck}} is a \sphinxstyleemphasis{yes/no} input (more information can be found here: {\hyperref[\detokenize{spellcheck:spellcheck-label}]{\sphinxcrossref{\DUrole{std,std-ref}{Spellchecking}}}})

\item {} 
\sphinxcode{\sphinxupquote{createLMS}} and \sphinxcode{\sphinxupquote{createLMS\_text}} are \sphinxstyleemphasis{yes/no} answers (see here: {\hyperref[\detokenize{lms:lms-label}]{\sphinxcrossref{\DUrole{std,std-ref}{LMS Output}}}})

\item {} 
\sphinxcode{\sphinxupquote{base}} is only used if the  \sphinxcode{\sphinxupquote{createLMS: yes}} is used. It is also described in {\hyperref[\detokenize{lms:lms-label}]{\sphinxcrossref{\DUrole{std,std-ref}{LMS Output}}}}.

\item {} 
\sphinxcode{\sphinxupquote{dir}} is only used if the  \sphinxcode{\sphinxupquote{createLMS: yes}} is used. It is also described in {\hyperref[\detokenize{lms:lms-label}]{\sphinxcrossref{\DUrole{std,std-ref}{LMS Output}}}}.

\item {} 
\sphinxcode{\sphinxupquote{title1}} and \sphinxcode{\sphinxupquote{title2}} are used to assemble the PDF files (both the raw exam and the version with answer keys when requested)

\item {} 
\sphinxcode{\sphinxupquote{solutionKey}} is a \sphinxstyleemphasis{yes/no} input. This provides the answer Key in a PDF file with highlighted answers (see here {\hyperref[\detokenize{pdf:pdf-label}]{\sphinxcrossref{\DUrole{std,std-ref}{Portable Data File creation}}}})

\end{enumerate}

\begin{sphinxadmonition}{note}{Note:}
This configuration file was used to create the examples shown in {\hyperref[\detokenize{Full Example:example}]{\sphinxcrossref{\DUrole{std,std-ref}{Example of multiple questions}}}}.
\end{sphinxadmonition}


\chapter{Portable Data File creation}
\label{\detokenize{pdf:portable-data-file-creation}}\label{\detokenize{pdf:pdf-label}}\label{\detokenize{pdf::doc}}
By default, \sphinxstyleemphasis{yaml2lms} always creates a PDF file, using latex processing and the \sphinxhref{http://www-math.mit.edu/~psh/exam/examdoc.pdf}{exam} class. The output file name (\sphinxstylestrong{\textless{}prefix\textgreater{}.pdf}) is built based on the \sphinxstyleemphasis{yaml} file name provided using \sphinxcode{\sphinxupquote{yamlfile: \textless{}prefix\textgreater{}.yaml}} in \sphinxstyleemphasis{config.yaml}.

The document will have a two\sphinxhyphen{}line title, provided by the tags \sphinxcode{\sphinxupquote{title1}} and \sphinxcode{\sphinxupquote{title2}} in the \sphinxstyleemphasis{config.yaml} file.

An example of PDF file output is shown in {\hyperref[\detokenize{Full Example:example}]{\sphinxcrossref{\DUrole{std,std-ref}{Example of multiple questions}}}}.

Two additional PDF files can be created if requested in the \sphinxstyleemphasis{config.yaml} files as described below.


\section{Answer Key PDF file}
\label{\detokenize{pdf:answer-key-pdf-file}}
If \sphinxcode{\sphinxupquote{solutionKey: yes}} is specified, in addition to the questionnaire itself, a file, named \sphinxstylestrong{\textless{}prefix\textgreater{}\_solutions.pdf} is created where the \sphinxstyleemphasis{correct} asnwers are highlighted. (See  {\hyperref[\detokenize{Full Example:example}]{\sphinxcrossref{\DUrole{std,std-ref}{Example of multiple questions}}}} for an example).


\section{Spellchecked PDF file}
\label{\detokenize{pdf:spellchecked-pdf-file}}
If \sphinxcode{\sphinxupquote{spellcheck: yes}} is specified, \sphinxstyleemphasis{yaml2.lms} performs a rudimentary spell checking of the input. This feature is described in details at {\hyperref[\detokenize{spellcheck:spellcheck-label}]{\sphinxcrossref{\DUrole{std,std-ref}{Spellchecking}}}}.

\begin{sphinxadmonition}{note}{Note:}
It is recommended to carefully check the PDF files before starting the process of creating LMS files. The \sphinxcode{\sphinxupquote{createLMS: yes}} option can take a few minutes or so to process, depending on the number of questions. And it is best to perform this once you have checked the spelling and the correctness of the answer keys. Note, also, that the spellchecking is time consuming and once you are done spellchecking, it is best to turn that option off.
\end{sphinxadmonition}


\chapter{LMS Output}
\label{\detokenize{lms:lms-output}}\label{\detokenize{lms:lms-label}}\label{\detokenize{lms::doc}}
One of the major goals of \sphinxstyleemphasis{yaml2lms} is to create an interface between a list of questions you create locally to a format that can be directly imported into an online teaching platform such as Blackboard. It is called Learning Management System (LMS) at Rensselaer Polytechnic Institute. However, the same format is used widely in systems such as \sphinxstyleemphasis{canvas}.

It is possible to import questions on LMS but the format is somewhat
difficult to handle. The format can be found at various places online,
such as as on the \sphinxhref{https://help.blackboard.com/Learn/Instructor/Tests\_Pools\_Surveys/Reuse\_Questions/Upload\_Questions/}{Blackboard} website.

The primary goal of this script is to translate the \sphinxstyleemphasis{yaml}\sphinxhyphen{}formatted questions into a file readable by Blackboard.


\section{Text only}
\label{\detokenize{lms:text-only}}
If you only have unformatted text in your questions, this is the option you need to use in the \sphinxstyleemphasis{config.yaml} file (\sphinxcode{\sphinxupquote{createLMS\_text: yes}}), as described in {\hyperref[\detokenize{usage:config-label}]{\sphinxcrossref{\DUrole{std,std-ref}{Basic usage and configuration file}}}}. In this case, all the information is stored in a file that can be directly uploaded onto lms.

\begin{sphinxadmonition}{note}{Note:}
the name of the output file is built from the \textless{}prefix\textgreater{} of the input file name provided in \sphinxstyleemphasis{config.yaml} under \sphinxcode{\sphinxupquote{yamlfile: \textless{}prefix\textgreater{}.yaml}} to which “\_LMS\_text.txt” is appended.
\end{sphinxadmonition}

Example:

If the input is:

\begin{sphinxVerbatim}[commandchars=\\\{\}]
\PYG{p+pIndicator}{\PYGZhy{}} \PYG{n+nt}{Type}\PYG{p}{:} \PYG{l+lScalar+lScalarPlain}{MC}
  \PYG{n+nt}{Text}\PYG{p}{:} \PYG{p+pIndicator}{\PYGZgt{}}\PYG{p+pIndicator}{\PYGZhy{}}
    \PYG{n+no}{Consider the exchange operator P12 whose effect is to swap two}
    \PYG{n+no}{particles. What is true among the assertions below?}
  \PYG{n+nt}{Answers}\PYG{p}{:}
  \PYG{p+pIndicator}{\PYGZhy{}} \PYG{n+nt}{Choice}\PYG{p}{:} \PYG{l+lScalar+lScalarPlain}{The}\PYG{l+lScalar+lScalarPlain}{ }\PYG{l+lScalar+lScalarPlain}{eigenvalues}\PYG{l+lScalar+lScalarPlain}{ }\PYG{l+lScalar+lScalarPlain}{of}\PYG{l+lScalar+lScalarPlain}{ }\PYG{l+lScalar+lScalarPlain}{the}\PYG{l+lScalar+lScalarPlain}{ }\PYG{l+lScalar+lScalarPlain}{operator}\PYG{l+lScalar+lScalarPlain}{ }\PYG{l+lScalar+lScalarPlain}{are}\PYG{l+lScalar+lScalarPlain}{ }\PYG{l+lScalar+lScalarPlain}{complex.}
  \PYG{p+pIndicator}{\PYGZhy{}} \PYG{n+nt}{Choice}\PYG{p}{:} \PYG{l+lScalar+lScalarPlain}{The}\PYG{l+lScalar+lScalarPlain}{ }\PYG{l+lScalar+lScalarPlain}{eigenvalues}\PYG{l+lScalar+lScalarPlain}{ }\PYG{l+lScalar+lScalarPlain}{of}\PYG{l+lScalar+lScalarPlain}{ }\PYG{l+lScalar+lScalarPlain}{the}\PYG{l+lScalar+lScalarPlain}{ }\PYG{l+lScalar+lScalarPlain}{operator}\PYG{l+lScalar+lScalarPlain}{ }\PYG{l+lScalar+lScalarPlain}{are}\PYG{l+lScalar+lScalarPlain}{ }\PYG{l+lScalar+lScalarPlain}{positive.}
  \PYG{p+pIndicator}{\PYGZhy{}} \PYG{n+nt}{Choice}\PYG{p}{:} \PYG{p+pIndicator}{\PYGZgt{}}\PYG{p+pIndicator}{\PYGZhy{}}
      \PYG{n+no}{The eigenvalues of the operator can only take two values for any}
      \PYG{n+no}{wavefunction describing a pair of Bosons or a pair of Fermions.}
    \PYG{n+nt}{Validity}\PYG{p}{:} \PYG{l+lScalar+lScalarPlain}{correct}
  \PYG{p+pIndicator}{\PYGZhy{}} \PYG{n+nt}{Choice}\PYG{p}{:} \PYG{l+s}{\PYGZsq{}}\PYG{l+s}{The}\PYG{n+nv}{ }\PYG{l+s}{operator}\PYG{n+nv}{ }\PYG{l+s}{is}\PYG{n+nv}{ }\PYG{l+s}{unitary}\PYG{n+nv}{ }\PYG{l+s}{but}\PYG{n+nv}{ }\PYG{l+s}{not}\PYG{n+nv}{ }\PYG{l+s}{Hermitian.}\PYG{l+s}{\PYGZsq{}}
  \PYG{p+pIndicator}{\PYGZhy{}} \PYG{n+nt}{Choice}\PYG{p}{:} \PYG{l+s}{\PYGZsq{}}\PYG{l+s}{None}\PYG{n+nv}{ }\PYG{l+s}{of}\PYG{n+nv}{ }\PYG{l+s}{the}\PYG{n+nv}{ }\PYG{l+s}{other}\PYG{n+nv}{ }\PYG{l+s}{claims}\PYG{n+nv}{ }\PYG{l+s}{are}\PYG{n+nv}{ }\PYG{l+s}{correct.}\PYG{n+nv}{  }\PYG{l+s}{\PYGZsq{}}
  \PYG{n+nt}{Note}\PYG{p}{:} \PYG{l+lScalar+lScalarPlain}{lecture}\PYG{l+lScalar+lScalarPlain}{ }\PYG{l+lScalar+lScalarPlain}{29}
\end{sphinxVerbatim}

The output becomes:
\begin{quote}

\begin{sphinxVerbatim}[commandchars=\\\{\}]
MC Consider the exchange operator P12 whose effect is
 to swap two particles. What is true among the assertions below?
 The eigenvalues of the operator are complex.  incorrect The
 eigenvalues of the operator are positive.  incorrect The
 eigenvalues of the operator can only take two values for any
 wavefunction describing a pair of Bosons or a pair of Fermions.
 correct The operator is unitary but not Hermitian.  incorrect
 None of the other claims are correct.  incorrect
\end{sphinxVerbatim}
\end{quote}

You can upload the file \sphinxcode{\sphinxupquote{\textless{}prefix\textgreater{}\_lms\_text.txt}} on LMS when creating a new test (or adding a question to an existing test) and you obtain a result like this:

\noindent\sphinxincludegraphics[width=800\sphinxpxdimen]{{LMS_text_export}.png}

\begin{sphinxadmonition}{note}{Note:}
In this example, we only used one question but if the \sphinxstyleemphasis{yaml} file contains multiple questions, they will be included into the test on LMS.
\end{sphinxadmonition}


\section{LMS using images}
\label{\detokenize{lms:lms-using-images}}
The method described above works very well and is very fast in terms of processing time.
However, it does not allow for fancy formatting and, while possible, the inclusion of math symbols is neither straitghforward nor totally satisfactory.
Here, I describe the second method to create an LMS test using more advanced formatting. In this case you need to use this option: \sphinxcode{\sphinxupquote{createLMS: yes}}.

It is \sphinxstylestrong{important to note} that this mode uses a collection of small images to assemble the questions (each image is created by \(\textrm{\LaTeX}\)). So, the only caveat with this method is that you need a place where you will copy the images and that place has to be accessible on the web.

\begin{sphinxadmonition}{important}{Important:}
The \sphinxstyleemphasis{LMS using images} mode is only possible if you have a place where you can copy the images to be accessible on a browser.
\end{sphinxadmonition}

The place where the images will be copied is provided in the keyword \sphinxcode{\sphinxupquote{base: \textless{}httpsite\textgreater{}}} provided in file \sphinxstyleemphasis{config.yaml}, as described in {\hyperref[\detokenize{usage:config-label}]{\sphinxcrossref{\DUrole{std,std-ref}{Basic usage and configuration file}}}}.

\begin{sphinxadmonition}{note}{Note:}
the name of the output file is built from the \textless{}prefix\textgreater{} of the input file name provided in \sphinxstyleemphasis{config.yaml} under \sphinxcode{\sphinxupquote{yamlfile: \textless{}prefix\textgreater{}.yaml}} to which “\_LMS\_png.txt” is appended.
\end{sphinxadmonition}

Let’s describe the process using an example. The example is similar to the one we use for the text\sphinxhyphen{}only option but in this case, we added some latex codes and also some formating code.

If the input is:

\begin{sphinxVerbatim}[commandchars=\\\{\}]
\PYG{p+pIndicator}{\PYGZhy{}} \PYG{n+nt}{Type}\PYG{p}{:} \PYG{l+lScalar+lScalarPlain}{MC}
  \PYG{n+nt}{Text}\PYG{p}{:} \PYG{p+pIndicator}{\PYGZgt{}}\PYG{p+pIndicator}{\PYGZhy{}}
    \PYG{n+no}{Consider the exchange operator \PYGZdl{}\PYGZbs{}hat\PYGZob{}P\PYGZcb{}\PYGZus{}\PYGZob{}12\PYGZcb{}\PYGZdl{}  whose effect is to swap two}
    \PYG{n+no}{particles. What is \PYGZbs{}textbf\PYGZob{}true\PYGZcb{} among the assertions below?}
  \PYG{n+nt}{Answers}\PYG{p}{:}
  \PYG{p+pIndicator}{\PYGZhy{}} \PYG{n+nt}{Choice}\PYG{p}{:} \PYG{l+lScalar+lScalarPlain}{The}\PYG{l+lScalar+lScalarPlain}{ }\PYG{l+lScalar+lScalarPlain}{eigenvalues}\PYG{l+lScalar+lScalarPlain}{ }\PYG{l+lScalar+lScalarPlain}{of}\PYG{l+lScalar+lScalarPlain}{ }\PYG{l+lScalar+lScalarPlain}{the}\PYG{l+lScalar+lScalarPlain}{ }\PYG{l+lScalar+lScalarPlain}{operator}\PYG{l+lScalar+lScalarPlain}{ }\PYG{l+lScalar+lScalarPlain}{are}\PYG{l+lScalar+lScalarPlain}{ }\PYG{l+lScalar+lScalarPlain}{\PYGZbs{}textit\PYGZob{}complex\PYGZcb{}.}
  \PYG{p+pIndicator}{\PYGZhy{}} \PYG{n+nt}{Choice}\PYG{p}{:} \PYG{l+lScalar+lScalarPlain}{The}\PYG{l+lScalar+lScalarPlain}{ }\PYG{l+lScalar+lScalarPlain}{eigenvalues}\PYG{l+lScalar+lScalarPlain}{ }\PYG{l+lScalar+lScalarPlain}{of}\PYG{l+lScalar+lScalarPlain}{ }\PYG{l+lScalar+lScalarPlain}{the}\PYG{l+lScalar+lScalarPlain}{ }\PYG{l+lScalar+lScalarPlain}{operator}\PYG{l+lScalar+lScalarPlain}{ }\PYG{l+lScalar+lScalarPlain}{are}\PYG{l+lScalar+lScalarPlain}{ }\PYG{l+lScalar+lScalarPlain}{positive.}
  \PYG{p+pIndicator}{\PYGZhy{}} \PYG{n+nt}{Choice}\PYG{p}{:} \PYG{p+pIndicator}{\PYGZgt{}}\PYG{p+pIndicator}{\PYGZhy{}}
      \PYG{n+no}{The eigenvalues of the operator can only take two values for any}
      \PYG{n+no}{wavefunction describing a pair of Bosons or a pair of Fermions.}
    \PYG{n+nt}{Validity}\PYG{p}{:} \PYG{l+lScalar+lScalarPlain}{correct}
  \PYG{p+pIndicator}{\PYGZhy{}} \PYG{n+nt}{Choice}\PYG{p}{:} \PYG{l+s}{\PYGZsq{}}\PYG{l+s}{The}\PYG{n+nv}{ }\PYG{l+s}{operator}\PYG{n+nv}{ }\PYG{l+s}{is}\PYG{n+nv}{ }\PYG{l+s}{unitary}\PYG{n+nv}{ }\PYG{l+s}{but}\PYG{n+nv}{ }\PYG{l+s}{not}\PYG{n+nv}{ }\PYG{l+s}{Hermitian.}\PYG{l+s}{\PYGZsq{}}
  \PYG{p+pIndicator}{\PYGZhy{}} \PYG{n+nt}{Choice}\PYG{p}{:} \PYG{l+s}{\PYGZsq{}}\PYG{l+s}{None}\PYG{n+nv}{ }\PYG{l+s}{of}\PYG{n+nv}{ }\PYG{l+s}{the}\PYG{n+nv}{ }\PYG{l+s}{other}\PYG{n+nv}{ }\PYG{l+s}{claims}\PYG{n+nv}{ }\PYG{l+s}{are}\PYG{n+nv}{ }\PYG{l+s}{correct.}\PYG{n+nv}{  }\PYG{l+s}{\PYGZsq{}}
  \PYG{n+nt}{Note}\PYG{p}{:} \PYG{l+lScalar+lScalarPlain}{lecture}\PYG{l+lScalar+lScalarPlain}{ }\PYG{l+lScalar+lScalarPlain}{29}
\end{sphinxVerbatim}

When using the option \sphinxcode{\sphinxupquote{createLMS: yes}}, \sphinxstyleemphasis{yaml2lms} creates a file \sphinxcode{\sphinxupquote{\textless{}prefix\textgreater{}\_lms\_png.txt}} along with a directory with a collection of images
The output stored in  \sphinxcode{\sphinxupquote{\textless{}prefix\textgreater{}\_lms\_png.tx}} looks something like this (though you will probably never have to look at it, as the idea of the script is to avoid it!)

\begin{sphinxVerbatim}[commandchars=\\\{\}]
\PYG{n}{MC} \PYG{o}{\PYGZlt{}}\PYG{n}{p}\PYG{o}{\PYGZgt{}}\PYG{o}{\PYGZlt{}}\PYG{n}{img}
\PYG{n}{src}\PYG{o}{=}\PYG{l+s+s2}{\PYGZdq{}}\PYG{l+s+s2}{http://homepages.rpi.edu/\PYGZti{}meuniv/Images/TSM\PYGZus{}F20/Questions\PYGZus{}THERMO\PYGZus{}test3/Q0/Q0.png}\PYG{l+s+s2}{\PYGZdq{}}
\PYG{n}{height}\PYG{o}{=}\PYG{l+s+s2}{\PYGZdq{}}\PYG{l+s+s2}{33}\PYG{l+s+s2}{\PYGZdq{}} \PYG{o}{/}\PYG{o}{\PYGZgt{}}\PYG{o}{\PYGZlt{}}\PYG{o}{/}\PYG{n}{p}\PYG{o}{\PYGZgt{}} \PYG{o}{\PYGZlt{}}\PYG{n}{p}\PYG{o}{\PYGZgt{}}\PYG{o}{\PYGZlt{}}\PYG{n}{img}
\PYG{n}{src}\PYG{o}{=}\PYG{l+s+s2}{\PYGZdq{}}\PYG{l+s+s2}{http://homepages.rpi.edu/\PYGZti{}meuniv/Images/TSM\PYGZus{}F20/Questions\PYGZus{}THERMO\PYGZus{}test3/Q0/Q0\PYGZus{}0.png}\PYG{l+s+s2}{\PYGZdq{}}
\PYG{n}{height}\PYG{o}{=}\PYG{l+s+s2}{\PYGZdq{}}\PYG{l+s+s2}{33}\PYG{l+s+s2}{\PYGZdq{}} \PYG{o}{/}\PYG{o}{\PYGZgt{}}\PYG{o}{\PYGZlt{}}\PYG{o}{/}\PYG{n}{p}\PYG{o}{\PYGZgt{}} \PYG{n}{incorrect} \PYG{o}{\PYGZlt{}}\PYG{n}{p}\PYG{o}{\PYGZgt{}}\PYG{o}{\PYGZlt{}}\PYG{n}{img}
\PYG{n}{src}\PYG{o}{=}\PYG{l+s+s2}{\PYGZdq{}}\PYG{l+s+s2}{http://homepages.rpi.edu/\PYGZti{}meuniv/Images/TSM\PYGZus{}F20/Questions\PYGZus{}THERMO\PYGZus{}test3/Q0/Q0\PYGZus{}1.png}\PYG{l+s+s2}{\PYGZdq{}}
\PYG{n}{height}\PYG{o}{=}\PYG{l+s+s2}{\PYGZdq{}}\PYG{l+s+s2}{33}\PYG{l+s+s2}{\PYGZdq{}} \PYG{o}{/}\PYG{o}{\PYGZgt{}}\PYG{o}{\PYGZlt{}}\PYG{o}{/}\PYG{n}{p}\PYG{o}{\PYGZgt{}} \PYG{n}{incorrect} \PYG{o}{\PYGZlt{}}\PYG{n}{p}\PYG{o}{\PYGZgt{}}\PYG{o}{\PYGZlt{}}\PYG{n}{img}
\PYG{n}{src}\PYG{o}{=}\PYG{l+s+s2}{\PYGZdq{}}\PYG{l+s+s2}{http://homepages.rpi.edu/\PYGZti{}meuniv/Images/TSM\PYGZus{}F20/Questions\PYGZus{}THERMO\PYGZus{}test3/Q0/Q0\PYGZus{}2.png}\PYG{l+s+s2}{\PYGZdq{}}
\PYG{n}{height}\PYG{o}{=}\PYG{l+s+s2}{\PYGZdq{}}\PYG{l+s+s2}{33}\PYG{l+s+s2}{\PYGZdq{}} \PYG{o}{/}\PYG{o}{\PYGZgt{}}\PYG{o}{\PYGZlt{}}\PYG{o}{/}\PYG{n}{p}\PYG{o}{\PYGZgt{}} \PYG{n}{correct} \PYG{o}{\PYGZlt{}}\PYG{n}{p}\PYG{o}{\PYGZgt{}}\PYG{o}{\PYGZlt{}}\PYG{n}{img}
\PYG{n}{src}\PYG{o}{=}\PYG{l+s+s2}{\PYGZdq{}}\PYG{l+s+s2}{http://homepages.rpi.edu/\PYGZti{}meuniv/Images/TSM\PYGZus{}F20/Questions\PYGZus{}THERMO\PYGZus{}test3/Q0/Q0\PYGZus{}3.png}\PYG{l+s+s2}{\PYGZdq{}}
\PYG{n}{height}\PYG{o}{=}\PYG{l+s+s2}{\PYGZdq{}}\PYG{l+s+s2}{33}\PYG{l+s+s2}{\PYGZdq{}} \PYG{o}{/}\PYG{o}{\PYGZgt{}}\PYG{o}{\PYGZlt{}}\PYG{o}{/}\PYG{n}{p}\PYG{o}{\PYGZgt{}} \PYG{n}{incorrect} \PYG{o}{\PYGZlt{}}\PYG{n}{p}\PYG{o}{\PYGZgt{}}\PYG{o}{\PYGZlt{}}\PYG{n}{img}
\PYG{n}{src}\PYG{o}{=}\PYG{l+s+s2}{\PYGZdq{}}\PYG{l+s+s2}{http://homepages.rpi.edu/\PYGZti{}meuniv/Images/TSM\PYGZus{}F20/Questions\PYGZus{}THERMO\PYGZus{}test3/Q0/Q0\PYGZus{}4.png}\PYG{l+s+s2}{\PYGZdq{}}
\PYG{n}{height}\PYG{o}{=}\PYG{l+s+s2}{\PYGZdq{}}\PYG{l+s+s2}{33}\PYG{l+s+s2}{\PYGZdq{}} \PYG{o}{/}\PYG{o}{\PYGZgt{}}\PYG{o}{\PYGZlt{}}\PYG{o}{/}\PYG{n}{p}\PYG{o}{\PYGZgt{}} \PYG{n}{incorrect}
\end{sphinxVerbatim}

You can see from the example that the script tells LMS that the various image files are stored, in this example, at “\sphinxurl{http://homepages.rpi.edu/~meuniv/Images/TSM\_F20/Questions\_THERMO\_test3}”.This address was created using the \sphinxcode{\sphinxupquote{base: "http://homepages.rpi.edu/\textasciitilde{}meuniv/Images/TSM\_F20/"}} provided in \sphinxstyleemphasis{config.yaml} and the directory is built from \sphinxcode{\sphinxupquote{dir: "THERMO"}}.

Now that you have completed this, you only need two more steps.
\begin{enumerate}
\sphinxsetlistlabels{\arabic}{enumi}{enumii}{}{.}%
\item {} 
Copy the image directory to the place where you want to move the file. This can be done easily with a method such as \sphinxcode{\sphinxupquote{scp \sphinxhyphen{}r Questions\_THERMO\_test3  meuniv@rcs.rpi.edu:\textasciitilde{}/public\_html/Images/TSM\_F20/.}} This line is provided for your convenience at the end of the script. Of course you need to change the username and the address of the server where you place the file. Here I use the space provided by my university but using a different repository may not be a bad idea (github or even dropbox).

\item {} 
Go to LMS and upload the questions, using the file \sphinxcode{\sphinxupquote{\textless{}prefix\textgreater{}\_lms\_png.txt}} as described above.

\end{enumerate}

After uploading the question, the exam looks like this:

\noindent\sphinxincludegraphics[width=800\sphinxpxdimen]{{LMS_png_export}.png}

The content is the same as in the example using text only. I personally prefer this approach as it makes for much nicer looking exam, even when no math is required. However, note that the other method is somewhat more straighforward.

\begin{sphinxadmonition}{important}{Important:}
Do not turn on the creation of LMS file (either methods) until you have carefully checked the PDF created with the default options. It is also important to check that the asnwers you selected as correct are indeed correct (check the PDF with the answer keys). It is always possible to change that on LMS itself but it is not as easy.
\end{sphinxadmonition}


\chapter{YAML Format (input)}
\label{\detokenize{Question format:yaml-format-input}}\label{\detokenize{Question format:my-reference-label}}\label{\detokenize{Question format::doc}}
Each question is provided in \sphinxtitleref{yaml} format. The format is somewhat unforgiving as spacings and alignments need to be correct for the file to be readable. Below is an example of simple question using this format. Each line will be described in details.


\section{Question example}
\label{\detokenize{Question format:question-example}}
\begin{sphinxVerbatim}[commandchars=\\\{\},numbers=left,firstnumber=1,stepnumber=1]
   \PYG{p+pIndicator}{\PYGZhy{}} \PYG{n+nt}{Type}\PYG{p}{:} \PYG{l+lScalar+lScalarPlain}{MC}
     \PYG{n+nt}{Text}\PYG{p}{:} \PYG{p+pIndicator}{\PYGZgt{}}\PYG{p+pIndicator}{\PYGZhy{}}
      \PYG{n+no}{In the screencast we derived an expression for the Fermi\PYGZhy{}Dirac distribution}
      \PYG{n+no}{using the grand canonical ensemble. What is the grand canonical ensemble?}
     \PYG{n+nt}{Size}\PYG{p}{:} \PYG{l+lScalar+lScalarPlain}{Auto}
     \PYG{n+nt}{Points}\PYG{p}{:} \PYG{l+lScalar+lScalarPlain}{2}
     \PYG{n+nt}{Answers}\PYG{p}{:}
      \PYG{p+pIndicator}{\PYGZhy{}} \PYG{n+nt}{Choice}\PYG{p}{:}  \PYG{p+pIndicator}{\PYGZgt{}}\PYG{p+pIndicator}{\PYGZhy{}}
          \PYG{n+no}{The ensemble of systems with fixed energy and entropy.}
        \PYG{n+nt}{Validity}\PYG{p}{:} \PYG{l+lScalar+lScalarPlain}{incorrect}
      \PYG{p+pIndicator}{\PYGZhy{}} \PYG{n+nt}{Choice}\PYG{p}{:} \PYG{p+pIndicator}{\PYGZgt{}}\PYG{p+pIndicator}{\PYGZhy{}}
          \PYG{n+no}{The ensemble of systems with fixed energy and chemical potential.}
        \PYG{n+nt}{Validity}\PYG{p}{:} \PYG{l+lScalar+lScalarPlain}{incorrect}
      \PYG{p+pIndicator}{\PYGZhy{}} \PYG{n+nt}{Choice}\PYG{p}{:} \PYG{l+s}{\PYGZsq{}}\PYG{l+s}{The}\PYG{n+nv}{ }\PYG{l+s}{ensemble}\PYG{n+nv}{ }\PYG{l+s}{of}\PYG{n+nv}{ }\PYG{l+s}{systems}\PYG{n+nv}{ }\PYG{l+s}{with}\PYG{n+nv}{ }\PYG{l+s}{fixed}\PYG{n+nv}{ }\PYG{l+s}{temperature}\PYG{n+nv}{ }\PYG{l+s}{and}\PYG{n+nv}{ }\PYG{l+s}{chemical}\PYG{n+nv}{ }\PYG{l+s}{potential.}\PYG{l+s}{\PYGZsq{}}
        \PYG{n+nt}{Validity}\PYG{p}{:} \PYG{l+lScalar+lScalarPlain}{correct}
      \PYG{n+nt}{Skip}\PYG{p}{:} \PYG{l+s}{\PYGZsq{}}\PYG{l+s}{no}\PYG{l+s}{\PYGZsq{}}
      \PYG{n+nt}{Note}\PYG{p}{:} \PYG{l+lScalar+lScalarPlain}{question}\PYG{l+lScalar+lScalarPlain}{ }\PYG{l+lScalar+lScalarPlain}{regarding}\PYG{l+lScalar+lScalarPlain}{ }\PYG{l+lScalar+lScalarPlain}{lecture}\PYG{l+lScalar+lScalarPlain}{ }\PYG{l+lScalar+lScalarPlain}{29}
\end{sphinxVerbatim}


\section{Anatomy of a question}
\label{\detokenize{Question format:anatomy-of-a-question}}
We will now review each line. Basically, all name that are followed by “:” is a key in a dictionary. If you wish to use a “:” in your questions or answers, you need to use quotation marks (or the \sphinxcode{\sphinxupquote{\textgreater{}\sphinxhyphen{}}} sign \textendash{} it is usually used for text with multiple lines).
\begin{enumerate}
\sphinxsetlistlabels{\arabic}{enumi}{enumii}{}{.}%
\item {} 
New question starts with the type.

\end{enumerate}

\begin{sphinxVerbatim}[commandchars=\\\{\}]
\PYG{p+pIndicator}{\PYGZhy{}} \PYG{n+nt}{Type}\PYG{p}{:} \PYG{l+lScalar+lScalarPlain}{MC}
\end{sphinxVerbatim}

Each new question starts with a \sphinxcode{\sphinxupquote{\sphinxhyphen{} Type:}} keyword. The options are: \sphinxcode{\sphinxupquote{MC}} (Multiple Choice), \sphinxcode{\sphinxupquote{MA}} (Multiple Answers),…
\begin{enumerate}
\sphinxsetlistlabels{\arabic}{enumi}{enumii}{}{.}%
\setcounter{enumi}{1}
\item {} 
Description of the question itself.

\end{enumerate}

\begin{sphinxVerbatim}[commandchars=\\\{\}]
\PYG{p+pIndicator}{\PYGZhy{}} \PYG{n+nt}{Text}\PYG{p}{:} \PYG{l+s}{\PYGZsq{}}\PYG{l+s}{This}\PYG{n+nv}{ }\PYG{l+s}{is}\PYG{n+nv}{ }\PYG{l+s}{my}\PYG{n+nv}{ }\PYG{l+s}{question}\PYG{l+s}{\PYGZsq{}}
\end{sphinxVerbatim}

or

\begin{sphinxVerbatim}[commandchars=\\\{\}]
\PYG{p+pIndicator}{\PYGZhy{}} \PYG{n+nt}{Text}\PYG{p}{:} \PYG{p+pIndicator}{\PYGZgt{}}\PYG{p+pIndicator}{\PYGZhy{}}
   \PYG{n+no}{This is a multiline question with various parts in it.}
   \PYG{n+no}{Try it if you want to.}
\end{sphinxVerbatim}

\begin{sphinxadmonition}{tip}{Tip:}
The advantage of this appraoch is that you can use Latex commands in both text and math modes in all the texts used in the yaml file.
\end{sphinxadmonition}
\begin{enumerate}
\sphinxsetlistlabels{\arabic}{enumi}{enumii}{}{.}%
\setcounter{enumi}{2}
\item {} 
For multiple\sphinxhyphen{}choice questions, you then have the list of possible answers:

\end{enumerate}

\begin{sphinxVerbatim}[commandchars=\\\{\}]
\PYG{n+nt}{Answers}\PYG{p}{:}
     \PYG{p+pIndicator}{\PYGZhy{}} \PYG{n+nt}{Choice}\PYG{p}{:}  \PYG{p+pIndicator}{\PYGZgt{}}\PYG{p+pIndicator}{\PYGZhy{}}
         \PYG{n+no}{The ensemble of systems with fixed energy and entropy.}
       \PYG{n+nt}{Validity}\PYG{p}{:} \PYG{l+lScalar+lScalarPlain}{incorrect}
\end{sphinxVerbatim}

Each Choice can be entered in a single\sphinxhyphen{}line or multiline format. There is a second keyword called \sphinxcode{\sphinxupquote{Validity}} (case sensitive) to assign an \sphinxtitleref{incorrect} or \sphinxtitleref{correct} attribute to the choice. Note that you only need to provide the information for a \sphinxstyleemphasis{correct} asnwer as the script will assign an \sphinxstyleemphasis{incorrect} attribute by default.
\begin{itemize}
\item {} 
Note 1: You can list as many \sphinxstyleemphasis{Choice} lines as you need.

\item {} 
Note 2: For some types of questions, only one answer can have the \sphinxcode{\sphinxupquote{Validity: correct}} attribute.

\end{itemize}
\begin{enumerate}
\sphinxsetlistlabels{\arabic}{enumi}{enumii}{}{.}%
\setcounter{enumi}{3}
\item {} 
You can skip the question from the file without deleting it by using: \sphinxcode{\sphinxupquote{Skip: \textquotesingle{}no\textquotesingle{}}} (this is an optional keyword)

\item {} 
To keep things tidy, you can add a note for each question, using the \sphinxcode{\sphinxupquote{Note:}} keyword.

\end{enumerate}

\begin{sphinxadmonition}{hint}{Hint:}
A good habit is to check your \sphinxtitleref{yaml} file using a free online
tool such as those provided by onlineyamltools.com (see, here: \sphinxurl{https://onlineyamltools.com/validate-yaml}). After a while you won’t make a mistake anymore but early on, this could be frustrating.
\end{sphinxadmonition}


\chapter{Spellchecking}
\label{\detokenize{spellcheck:spellchecking}}\label{\detokenize{spellcheck:spellcheck-label}}\label{\detokenize{spellcheck::doc}}
\begin{sphinxadmonition}{note}{Note:}
This features remains in development but can be useful to identify basic typos.
\end{sphinxadmonition}


\section{Usage}
\label{\detokenize{spellcheck:usage}}
When you use the \sphinxcode{\sphinxupquote{spellcheck: yes}} in the \sphinxcode{\sphinxupquote{config.yaml}} file, \sphinxstyleemphasis{yam2lms} will perform a spellcheck of the questions. Currently, the script uses the python library provided in the \sphinxcode{\sphinxupquote{spellchecker}} module.

The script will check all words in the document and provide a corrected version called \sphinxcode{\sphinxupquote{XXX\_SPELLCHECKED.pdf}} file. Using the example provided in {\hyperref[\detokenize{Question format:my-reference-label}]{\sphinxcrossref{\DUrole{std,std-ref}{YAML Format (input)}}}}, we get the result shown below.


\section{Notes}
\label{\detokenize{spellcheck:notes}}\begin{enumerate}
\sphinxsetlistlabels{\arabic}{enumi}{enumii}{}{.}%
\item {} 
The process is slow. Once you have run this and corrected your mistake, turn this off to avoid lenghty processing.

\item {} 
\sphinxstyleemphasis{yaml2lms} does not make any correction; instead it makes suggestions as shown in the exmaple below. In this example, most of the mistakes found are actually \sphinxstylestrong{not} mistakes and it is expected each user will look into this separately. Only one word was correctly flagged as incorrect.

\end{enumerate}


\section{Example}
\label{\detokenize{spellcheck:example}}
\noindent\sphinxincludegraphics[width=600\sphinxpxdimen]{{quiz20SPELLCHECKED}.png}


\chapter{Example of multiple questions}
\label{\detokenize{Full Example:example-of-multiple-questions}}\label{\detokenize{Full Example:example}}\label{\detokenize{Full Example::doc}}

\section{Yaml input}
\label{\detokenize{Full Example:yaml-input}}
\begin{sphinxVerbatim}[commandchars=\\\{\}]
\PYG{p+pIndicator}{\PYGZhy{}} \PYG{n+nt}{Type}\PYG{p}{:} \PYG{l+lScalar+lScalarPlain}{MC}
  \PYG{n+nt}{Text}\PYG{p}{:} \PYG{l+s}{\PYGZsq{}}\PYG{l+s}{Among}\PYG{n+nv}{ }\PYG{l+s}{the}\PYG{n+nv}{ }\PYG{l+s}{following}\PYG{n+nv}{ }\PYG{l+s}{claims,}\PYG{n+nv}{ }\PYG{l+s}{which}\PYG{n+nv}{ }\PYG{l+s}{one}\PYG{n+nv}{ }\PYG{l+s}{is}\PYG{n+nv}{ }\PYG{l+s}{\PYGZbs{}textbf\PYGZob{}not\PYGZcb{}}\PYG{n+nv}{ }\PYG{l+s}{true?}\PYG{l+s}{\PYGZsq{}}
  \PYG{n+nt}{Size}\PYG{p}{:} \PYG{l+lScalar+lScalarPlain}{Auto}
  \PYG{n+nt}{Points}\PYG{p}{:} \PYG{l+lScalar+lScalarPlain}{2}
  \PYG{n+nt}{Answers}\PYG{p}{:}
  \PYG{p+pIndicator}{\PYGZhy{}} \PYG{n+nt}{Choice}\PYG{p}{:} \PYG{l+lScalar+lScalarPlain}{Bosons}\PYG{l+lScalar+lScalarPlain}{ }\PYG{l+lScalar+lScalarPlain}{have}\PYG{l+lScalar+lScalarPlain}{ }\PYG{l+lScalar+lScalarPlain}{an}\PYG{l+lScalar+lScalarPlain}{ }\PYG{l+lScalar+lScalarPlain}{integer}\PYG{l+lScalar+lScalarPlain}{ }\PYG{l+lScalar+lScalarPlain}{spin}\PYG{l+lScalar+lScalarPlain}{ }\PYG{l+lScalar+lScalarPlain}{value}\PYG{l+lScalar+lScalarPlain}{ }\PYG{l+lScalar+lScalarPlain}{and}\PYG{l+lScalar+lScalarPlain}{ }\PYG{l+lScalar+lScalarPlain}{Fermions}\PYG{l+lScalar+lScalarPlain}{ }\PYG{l+lScalar+lScalarPlain}{have}\PYG{l+lScalar+lScalarPlain}{ }\PYG{l+lScalar+lScalarPlain}{an}\PYG{l+lScalar+lScalarPlain}{ }\PYG{l+lScalar+lScalarPlain}{half}\PYG{l+lScalar+lScalarPlain}{ }\PYG{l+lScalar+lScalarPlain}{integer}\PYG{l+lScalar+lScalarPlain}{ }\PYG{l+lScalar+lScalarPlain}{spin}\PYG{l+lScalar+lScalarPlain}{ }\PYG{l+lScalar+lScalarPlain}{value.}
  \PYG{p+pIndicator}{\PYGZhy{}} \PYG{n+nt}{Choice}\PYG{p}{:} \PYG{l+lScalar+lScalarPlain}{Bosons}\PYG{l+lScalar+lScalarPlain}{ }\PYG{l+lScalar+lScalarPlain}{and}\PYG{l+lScalar+lScalarPlain}{ }\PYG{l+lScalar+lScalarPlain}{Fermions}\PYG{l+lScalar+lScalarPlain}{ }\PYG{l+lScalar+lScalarPlain}{are}\PYG{l+lScalar+lScalarPlain}{ }\PYG{l+lScalar+lScalarPlain}{treated}\PYG{l+lScalar+lScalarPlain}{ }\PYG{l+lScalar+lScalarPlain}{as}\PYG{l+lScalar+lScalarPlain}{ }\PYG{l+lScalar+lScalarPlain}{indistiguishable}\PYG{l+lScalar+lScalarPlain}{ }\PYG{l+lScalar+lScalarPlain}{particles.}
  \PYG{p+pIndicator}{\PYGZhy{}} \PYG{n+nt}{Choice}\PYG{p}{:} \PYG{l+lScalar+lScalarPlain}{Multiple}\PYG{l+lScalar+lScalarPlain}{ }\PYG{l+lScalar+lScalarPlain}{Fermions}\PYG{l+lScalar+lScalarPlain}{ }\PYG{l+lScalar+lScalarPlain}{can}\PYG{l+lScalar+lScalarPlain}{ }\PYG{l+lScalar+lScalarPlain}{occupy}\PYG{l+lScalar+lScalarPlain}{ }\PYG{l+lScalar+lScalarPlain}{the}\PYG{l+lScalar+lScalarPlain}{ }\PYG{l+lScalar+lScalarPlain}{same}\PYG{l+lScalar+lScalarPlain}{ }\PYG{l+lScalar+lScalarPlain}{quantum}\PYG{l+lScalar+lScalarPlain}{ }\PYG{l+lScalar+lScalarPlain}{state.}
    \PYG{n+nt}{Validity}\PYG{p}{:} \PYG{l+lScalar+lScalarPlain}{correct}
  \PYG{p+pIndicator}{\PYGZhy{}} \PYG{n+nt}{Choice}\PYG{p}{:} \PYG{l+lScalar+lScalarPlain}{Multiple}\PYG{l+lScalar+lScalarPlain}{ }\PYG{l+lScalar+lScalarPlain}{Bosons}\PYG{l+lScalar+lScalarPlain}{ }\PYG{l+lScalar+lScalarPlain}{can}\PYG{l+lScalar+lScalarPlain}{ }\PYG{l+lScalar+lScalarPlain}{occupy}\PYG{l+lScalar+lScalarPlain}{ }\PYG{l+lScalar+lScalarPlain}{the}\PYG{l+lScalar+lScalarPlain}{ }\PYG{l+lScalar+lScalarPlain}{same}\PYG{l+lScalar+lScalarPlain}{ }\PYG{l+lScalar+lScalarPlain}{quantum}\PYG{l+lScalar+lScalarPlain}{ }\PYG{l+lScalar+lScalarPlain}{state.}
  \PYG{p+pIndicator}{\PYGZhy{}} \PYG{n+nt}{Choice}\PYG{p}{:} \PYG{l+s}{\PYGZsq{}}\PYG{l+s}{All}\PYG{n+nv}{ }\PYG{l+s}{the}\PYG{n+nv}{ }\PYG{l+s}{other}\PYG{n+nv}{ }\PYG{l+s}{claims}\PYG{n+nv}{ }\PYG{l+s}{are}\PYG{n+nv}{ }\PYG{l+s}{correct.}\PYG{n+nv}{  }\PYG{l+s}{\PYGZsq{}}
  \PYG{n+nt}{Note}\PYG{p}{:} \PYG{l+lScalar+lScalarPlain}{lecture}\PYG{l+lScalar+lScalarPlain}{ }\PYG{l+lScalar+lScalarPlain}{29}
\PYG{p+pIndicator}{\PYGZhy{}} \PYG{n+nt}{Type}\PYG{p}{:} \PYG{l+lScalar+lScalarPlain}{MC}
  \PYG{n+nt}{Text}\PYG{p}{:} \PYG{p+pIndicator}{\PYGZgt{}}\PYG{p+pIndicator}{\PYGZhy{}}
     \PYG{n+no}{Consider the exchange operator \PYGZdl{}\PYGZbs{}hat\PYGZob{}P\PYGZcb{}\PYGZus{}\PYGZob{}12\PYGZcb{}\PYGZdl{} whose effect is to swap two}
     \PYG{n+no}{particles. What is true among the assertions below?}
  \PYG{n+nt}{Size}\PYG{p}{:} \PYG{l+lScalar+lScalarPlain}{Auto}
  \PYG{n+nt}{Points}\PYG{p}{:} \PYG{l+lScalar+lScalarPlain}{2}
  \PYG{n+nt}{Answers}\PYG{p}{:}
  \PYG{p+pIndicator}{\PYGZhy{}} \PYG{n+nt}{Choice}\PYG{p}{:} \PYG{l+lScalar+lScalarPlain}{The}\PYG{l+lScalar+lScalarPlain}{ }\PYG{l+lScalar+lScalarPlain}{eigenvalues}\PYG{l+lScalar+lScalarPlain}{ }\PYG{l+lScalar+lScalarPlain}{of}\PYG{l+lScalar+lScalarPlain}{ }\PYG{l+lScalar+lScalarPlain}{the}\PYG{l+lScalar+lScalarPlain}{ }\PYG{l+lScalar+lScalarPlain}{operator}\PYG{l+lScalar+lScalarPlain}{ }\PYG{l+lScalar+lScalarPlain}{are}\PYG{l+lScalar+lScalarPlain}{ }\PYG{l+lScalar+lScalarPlain}{complex.}
  \PYG{p+pIndicator}{\PYGZhy{}} \PYG{n+nt}{Choice}\PYG{p}{:} \PYG{l+lScalar+lScalarPlain}{The}\PYG{l+lScalar+lScalarPlain}{ }\PYG{l+lScalar+lScalarPlain}{eigenvalues}\PYG{l+lScalar+lScalarPlain}{ }\PYG{l+lScalar+lScalarPlain}{of}\PYG{l+lScalar+lScalarPlain}{ }\PYG{l+lScalar+lScalarPlain}{the}\PYG{l+lScalar+lScalarPlain}{ }\PYG{l+lScalar+lScalarPlain}{operator}\PYG{l+lScalar+lScalarPlain}{ }\PYG{l+lScalar+lScalarPlain}{are}\PYG{l+lScalar+lScalarPlain}{ }\PYG{l+lScalar+lScalarPlain}{positive.}
  \PYG{p+pIndicator}{\PYGZhy{}} \PYG{n+nt}{Choice}\PYG{p}{:} \PYG{p+pIndicator}{\PYGZgt{}}\PYG{p+pIndicator}{\PYGZhy{}}
      \PYG{n+no}{The eigenvalues of the operator can only take two values for any}
      \PYG{n+no}{wavefunction describing a pair of Bosons or a pair of Fermions.}
    \PYG{n+nt}{Validity}\PYG{p}{:} \PYG{l+lScalar+lScalarPlain}{correct}
  \PYG{p+pIndicator}{\PYGZhy{}} \PYG{n+nt}{Choice}\PYG{p}{:} \PYG{l+s}{\PYGZsq{}}\PYG{l+s}{The}\PYG{n+nv}{ }\PYG{l+s}{operator}\PYG{n+nv}{ }\PYG{l+s}{is}\PYG{n+nv}{ }\PYG{l+s}{unitary}\PYG{n+nv}{ }\PYG{l+s}{but}\PYG{n+nv}{ }\PYG{l+s}{not}\PYG{n+nv}{ }\PYG{l+s}{Hermitian.}\PYG{l+s}{\PYGZsq{}}
  \PYG{p+pIndicator}{\PYGZhy{}} \PYG{n+nt}{Choice}\PYG{p}{:} \PYG{l+s}{\PYGZsq{}}\PYG{l+s}{None}\PYG{n+nv}{ }\PYG{l+s}{of}\PYG{n+nv}{ }\PYG{l+s}{the}\PYG{n+nv}{ }\PYG{l+s}{other}\PYG{n+nv}{ }\PYG{l+s}{claims}\PYG{n+nv}{ }\PYG{l+s}{are}\PYG{n+nv}{ }\PYG{l+s}{correct.}\PYG{n+nv}{  }\PYG{l+s}{\PYGZsq{}}
  \PYG{n+nt}{Note}\PYG{p}{:} \PYG{l+lScalar+lScalarPlain}{lecture}\PYG{l+lScalar+lScalarPlain}{ }\PYG{l+lScalar+lScalarPlain}{29}
\PYG{p+pIndicator}{\PYGZhy{}} \PYG{n+nt}{Type}\PYG{p}{:} \PYG{l+lScalar+lScalarPlain}{MC}
  \PYG{n+nt}{Text}\PYG{p}{:} \PYG{l+s}{\PYGZsq{}}\PYG{l+s}{At}\PYG{n+nv}{ }\PYG{l+s}{low}\PYG{n+nv}{ }\PYG{l+s}{energy,}\PYG{n+nv}{ }\PYG{l+s}{Fermions}\PYG{n+nv}{ }\PYG{l+s}{and}\PYG{n+nv}{ }\PYG{l+s}{Bosons}\PYG{n+nv}{ }\PYG{l+s}{follow}\PYG{n+nv}{ }\PYG{l+s}{the}\PYG{n+nv}{ }\PYG{l+s}{same}\PYG{n+nv}{ }\PYG{l+s}{statistical}\PYG{n+nv}{ }\PYG{l+s}{distribution.}\PYG{l+s}{\PYGZsq{}}
  \PYG{n+nt}{Size}\PYG{p}{:} \PYG{l+lScalar+lScalarPlain}{Auto}
  \PYG{n+nt}{Points}\PYG{p}{:} \PYG{l+lScalar+lScalarPlain}{2}
  \PYG{n+nt}{Answers}\PYG{p}{:}
  \PYG{p+pIndicator}{\PYGZhy{}} \PYG{n+nt}{Choice}\PYG{p}{:} \PYG{l+lScalar+lScalarPlain}{This}\PYG{l+lScalar+lScalarPlain}{ }\PYG{l+lScalar+lScalarPlain}{is}\PYG{l+lScalar+lScalarPlain}{ }\PYG{l+lScalar+lScalarPlain}{always}\PYG{l+lScalar+lScalarPlain}{ }\PYG{l+lScalar+lScalarPlain}{true.}
  \PYG{p+pIndicator}{\PYGZhy{}} \PYG{n+nt}{Choice}\PYG{p}{:} \PYG{l+lScalar+lScalarPlain}{This}\PYG{l+lScalar+lScalarPlain}{ }\PYG{l+lScalar+lScalarPlain}{is}\PYG{l+lScalar+lScalarPlain}{ }\PYG{l+lScalar+lScalarPlain}{sometimes}\PYG{l+lScalar+lScalarPlain}{ }\PYG{l+lScalar+lScalarPlain}{true.}
  \PYG{p+pIndicator}{\PYGZhy{}} \PYG{n+nt}{Choice}\PYG{p}{:} \PYG{l+lScalar+lScalarPlain}{This}\PYG{l+lScalar+lScalarPlain}{ }\PYG{l+lScalar+lScalarPlain}{is}\PYG{l+lScalar+lScalarPlain}{ }\PYG{l+lScalar+lScalarPlain}{never}\PYG{l+lScalar+lScalarPlain}{ }\PYG{l+lScalar+lScalarPlain}{true.}
    \PYG{n+nt}{Validity}\PYG{p}{:} \PYG{l+lScalar+lScalarPlain}{correct}
  \PYG{n+nt}{Note}\PYG{p}{:} \PYG{l+lScalar+lScalarPlain}{lecture}\PYG{l+lScalar+lScalarPlain}{ }\PYG{l+lScalar+lScalarPlain}{29}
\PYG{p+pIndicator}{\PYGZhy{}} \PYG{n+nt}{Type}\PYG{p}{:} \PYG{l+lScalar+lScalarPlain}{MC}
  \PYG{n+nt}{Text}\PYG{p}{:} \PYG{l+lScalar+lScalarPlain}{What}\PYG{l+lScalar+lScalarPlain}{ }\PYG{l+lScalar+lScalarPlain}{can}\PYG{l+lScalar+lScalarPlain}{ }\PYG{l+lScalar+lScalarPlain}{you}\PYG{l+lScalar+lScalarPlain}{ }\PYG{l+lScalar+lScalarPlain}{say}\PYG{l+lScalar+lScalarPlain}{ }\PYG{l+lScalar+lScalarPlain}{about}\PYG{l+lScalar+lScalarPlain}{ }\PYG{l+lScalar+lScalarPlain}{the}\PYG{l+lScalar+lScalarPlain}{ }\PYG{l+lScalar+lScalarPlain}{chemical}\PYG{l+lScalar+lScalarPlain}{ }\PYG{l+lScalar+lScalarPlain}{potential}\PYG{l+lScalar+lScalarPlain}{ }\PYG{l+lScalar+lScalarPlain}{(\PYGZdl{}\PYGZbs{}mu\PYGZdl{})?}
  \PYG{n+nt}{Size}\PYG{p}{:} \PYG{l+lScalar+lScalarPlain}{Auto}
  \PYG{n+nt}{Points}\PYG{p}{:} \PYG{l+lScalar+lScalarPlain}{2}
  \PYG{n+nt}{Answers}\PYG{p}{:}
  \PYG{p+pIndicator}{\PYGZhy{}} \PYG{n+nt}{Choice}\PYG{p}{:} \PYG{l+lScalar+lScalarPlain}{It}\PYG{l+lScalar+lScalarPlain}{ }\PYG{l+lScalar+lScalarPlain}{is}\PYG{l+lScalar+lScalarPlain}{ }\PYG{l+lScalar+lScalarPlain}{small}\PYG{l+lScalar+lScalarPlain}{ }\PYG{l+lScalar+lScalarPlain}{when}\PYG{l+lScalar+lScalarPlain}{ }\PYG{l+lScalar+lScalarPlain}{the}\PYG{l+lScalar+lScalarPlain}{ }\PYG{l+lScalar+lScalarPlain}{density}\PYG{l+lScalar+lScalarPlain}{ }\PYG{l+lScalar+lScalarPlain}{of}\PYG{l+lScalar+lScalarPlain}{ }\PYG{l+lScalar+lScalarPlain}{matter}\PYG{l+lScalar+lScalarPlain}{ }\PYG{l+lScalar+lScalarPlain}{(\PYGZdl{}n\PYGZdl{})}\PYG{l+lScalar+lScalarPlain}{ }\PYG{l+lScalar+lScalarPlain}{is}\PYG{l+lScalar+lScalarPlain}{ }\PYG{l+lScalar+lScalarPlain}{small.}
    \PYG{n+nt}{Validity}\PYG{p}{:} \PYG{l+lScalar+lScalarPlain}{correct}
  \PYG{p+pIndicator}{\PYGZhy{}} \PYG{n+nt}{Choice}\PYG{p}{:} \PYG{l+lScalar+lScalarPlain}{It}\PYG{l+lScalar+lScalarPlain}{ }\PYG{l+lScalar+lScalarPlain}{is}\PYG{l+lScalar+lScalarPlain}{ }\PYG{l+lScalar+lScalarPlain}{small}\PYG{l+lScalar+lScalarPlain}{ }\PYG{l+lScalar+lScalarPlain}{when}\PYG{l+lScalar+lScalarPlain}{ }\PYG{l+lScalar+lScalarPlain}{the}\PYG{l+lScalar+lScalarPlain}{ }\PYG{l+lScalar+lScalarPlain}{density}\PYG{l+lScalar+lScalarPlain}{ }\PYG{l+lScalar+lScalarPlain}{of}\PYG{l+lScalar+lScalarPlain}{ }\PYG{l+lScalar+lScalarPlain}{matter}\PYG{l+lScalar+lScalarPlain}{ }\PYG{l+lScalar+lScalarPlain}{(\PYGZdl{}n\PYGZdl{})}\PYG{l+lScalar+lScalarPlain}{ }\PYG{l+lScalar+lScalarPlain}{is}\PYG{l+lScalar+lScalarPlain}{ }\PYG{l+lScalar+lScalarPlain}{large.}
  \PYG{p+pIndicator}{\PYGZhy{}} \PYG{n+nt}{Choice}\PYG{p}{:} \PYG{l+s}{\PYGZsq{}}\PYG{l+s}{It}\PYG{n+nv}{ }\PYG{l+s}{does}\PYG{n+nv}{ }\PYG{l+s}{not}\PYG{n+nv}{ }\PYG{l+s}{depend}\PYG{n+nv}{ }\PYG{l+s}{on}\PYG{n+nv}{ }\PYG{l+s}{the}\PYG{n+nv}{ }\PYG{l+s}{density}\PYG{n+nv}{ }\PYG{l+s}{of}\PYG{n+nv}{ }\PYG{l+s}{matter}\PYG{n+nv}{ }\PYG{l+s}{(\PYGZdl{}n\PYGZdl{}).}\PYG{n+nv}{  }\PYG{l+s}{\PYGZsq{}}
  \PYG{n+nt}{Note}\PYG{p}{:} \PYG{l+lScalar+lScalarPlain}{lecture}\PYG{l+lScalar+lScalarPlain}{ }\PYG{l+lScalar+lScalarPlain}{29}
\PYG{p+pIndicator}{\PYGZhy{}} \PYG{n+nt}{Type}\PYG{p}{:} \PYG{l+lScalar+lScalarPlain}{MC}
  \PYG{n+nt}{Text}\PYG{p}{:} \PYG{p+pIndicator}{\PYGZgt{}}\PYG{p+pIndicator}{\PYGZhy{}}
   \PYG{n+no}{At high energy (large \PYGZdl{}E\PYGZhy{}\PYGZbs{}mu\PYGZdl{} values), we can use the Botzmann,}
   \PYG{n+no}{Fermi\PYGZhy{}Dirac, or Bose\PYGZhy{}Einstein disributrions to study any gas of}
   \PYG{n+no}{particles (e.g., photons, electrons,\PYGZbs{}ldots)}
  \PYG{n+nt}{Size}\PYG{p}{:} \PYG{l+lScalar+lScalarPlain}{Auto}
  \PYG{n+nt}{Points}\PYG{p}{:} \PYG{l+lScalar+lScalarPlain}{2}
  \PYG{n+nt}{Answers}\PYG{p}{:}
  \PYG{p+pIndicator}{\PYGZhy{}} \PYG{n+nt}{Choice}\PYG{p}{:} \PYG{l+s}{\PYGZsq{}}\PYG{l+s}{This}\PYG{n+nv}{ }\PYG{l+s}{is}\PYG{n+nv}{ }\PYG{l+s}{true}\PYG{n+nv}{ }\PYG{l+s}{because}\PYG{n+nv}{ }\PYG{l+s}{this}\PYG{n+nv}{ }\PYG{l+s}{corresponds}\PYG{n+nv}{ }\PYG{l+s}{to}\PYG{n+nv}{ }\PYG{l+s}{the}\PYG{n+nv}{ }\PYG{l+s}{low}\PYG{n+nv}{ }\PYG{l+s}{density}\PYG{n+nv}{ }\PYG{l+s}{limit.}\PYG{n+nv}{ }\PYG{l+s}{\PYGZsq{}}
    \PYG{n+nt}{Validity}\PYG{p}{:} \PYG{l+lScalar+lScalarPlain}{correct}
  \PYG{p+pIndicator}{\PYGZhy{}} \PYG{n+nt}{Choice}\PYG{p}{:} \PYG{p+pIndicator}{\PYGZgt{}}\PYG{p+pIndicator}{\PYGZhy{}}
      \PYG{n+no}{This is false because the three distributions do not converge to  one}
      \PYG{n+no}{another at high energy.}
  \PYG{p+pIndicator}{\PYGZhy{}} \PYG{n+nt}{Choice}\PYG{p}{:} \PYG{p+pIndicator}{\PYGZgt{}}\PYG{p+pIndicator}{\PYGZhy{}}
      \PYG{n+no}{This is a very good question. I wil make sure to read page 598 of  the}
      \PYG{n+no}{slides posted on SLACK to understand the answer.}
  \PYG{n+nt}{Note}\PYG{p}{:} \PYG{l+lScalar+lScalarPlain}{lecture}\PYG{l+lScalar+lScalarPlain}{ }\PYG{l+lScalar+lScalarPlain}{29}
\PYG{p+pIndicator}{\PYGZhy{}} \PYG{n+nt}{Type}\PYG{p}{:} \PYG{l+lScalar+lScalarPlain}{MC}
  \PYG{n+nt}{Text}\PYG{p}{:} \PYG{p+pIndicator}{\PYGZgt{}}\PYG{p+pIndicator}{\PYGZhy{}}
   \PYG{n+no}{Consider a bosonic particle extracted from a large distribution of the  same}
   \PYG{n+no}{particles. A specific measurement shows that its energy is 1\PYGZti{}eV  below the}
   \PYG{n+no}{chemical potential of the distribution.  What can you conclude?}
  \PYG{n+nt}{Size}\PYG{p}{:} \PYG{l+lScalar+lScalarPlain}{Auto}
  \PYG{n+nt}{Points}\PYG{p}{:} \PYG{l+lScalar+lScalarPlain}{2}
  \PYG{n+nt}{Answers}\PYG{p}{:}
  \PYG{p+pIndicator}{\PYGZhy{}} \PYG{n+nt}{Choice}\PYG{p}{:} \PYG{l+s}{\PYGZsq{}}\PYG{l+s}{The}\PYG{n+nv}{ }\PYG{l+s}{particle}\PYG{n+nv}{ }\PYG{l+s}{is}\PYG{n+nv}{ }\PYG{l+s}{very}\PYG{n+nv}{ }\PYG{l+s}{stable.}\PYG{n+nv}{ }\PYG{l+s}{\PYGZsq{}}
    \PYG{n+nt}{Validity}\PYG{p}{:} \PYG{l+lScalar+lScalarPlain}{incorrect}
  \PYG{p+pIndicator}{\PYGZhy{}} \PYG{n+nt}{Choice}\PYG{p}{:} \PYG{l+s}{\PYGZsq{}}\PYG{l+s}{The}\PYG{n+nv}{ }\PYG{l+s}{entropy}\PYG{n+nv}{ }\PYG{l+s}{of}\PYG{n+nv}{ }\PYG{l+s}{the}\PYG{n+nv}{ }\PYG{l+s}{particle}\PYG{n+nv}{ }\PYG{l+s}{is}\PYG{n+nv}{ }\PYG{l+s}{very}\PYG{n+nv}{ }\PYG{l+s}{small.}\PYG{n+nv}{ }\PYG{l+s}{\PYGZsq{}}
    \PYG{n+nt}{Validity}\PYG{p}{:} \PYG{l+lScalar+lScalarPlain}{incorrect}
  \PYG{p+pIndicator}{\PYGZhy{}} \PYG{n+nt}{Choice}\PYG{p}{:} \PYG{p+pIndicator}{\PYGZgt{}}\PYG{p+pIndicator}{\PYGZhy{}}
     \PYG{n+no}{Researchers who performed the measurements should sign up for  PHYS\PYGZhy{}2350}
     \PYG{n+no}{at RPI: their measurement is clearly wrong.}
    \PYG{n+nt}{Validity}\PYG{p}{:} \PYG{l+lScalar+lScalarPlain}{correct}
  \PYG{n+nt}{Note}\PYG{p}{:} \PYG{l+lScalar+lScalarPlain}{lecture}\PYG{l+lScalar+lScalarPlain}{ }\PYG{l+lScalar+lScalarPlain}{29}
\end{sphinxVerbatim}


\section{Latex Output}
\label{\detokenize{Full Example:latex-output}}
The latex output for this entry is thus:

\noindent\sphinxincludegraphics[width=600\sphinxpxdimen]{{quiz20}.png}


\section{Latex Output with keys}
\label{\detokenize{Full Example:latex-output-with-keys}}
If you used the \sphinxcode{\sphinxupquote{SolutionKey: yes}} option, you would get:

\noindent\sphinxincludegraphics[width=600\sphinxpxdimen]{{quiz20_solutions}.png}


\chapter{Work in Progress and Future Utilities:}
\label{\detokenize{WorkinProgress:work-in-progress-and-future-utilities}}\label{\detokenize{WorkinProgress::doc}}
Yaml2LMS is a hack. It is not perfect but it gets the job done in many cases.

THere are many things that can be improved (send your suggestions to \sphinxhref{mailto:vinmeunier@gmail.com}{me}).

Here is a list of things I’m working on:
\begin{enumerate}
\sphinxsetlistlabels{\arabic}{enumi}{enumii}{}{)}%
\item {} 
Improved spellchecker

\item {} 
Better management of latex packages

\end{enumerate}



\renewcommand{\indexname}{Index}
\printindex
\end{document}